\documentclass[%singlesided,
doublesided,
paper=a4,
fontsize=10pt
]{my-resume}


%%%%%%%%%%%%%%%%%%%%%%%%%%%%%%%%%%%%%%%%%%%%%%%%%%%%%%%%%%%%%%%%%%%%%%%%%%%%%%%%
% set geometry
%%%%%%%%%%%%%%%%%%%%%%%%%%%%%%%%%%%%%%%%%%%%%%%%%%%%%%%%%%%%%%%%%%%%%%%%%%%%%%%%

\setlength\highlightwidth{8cm}
\setlength\headerheight{4cm}            % note that margintop gets added to this value, i.e. the header bar is 5cm
\setlength\marginleft{1cm}
\setlength\marginright{\marginleft}      % needs to be 1.5 times to be actually equal. why?
\setlength\margintop{1cm}
\setlength\marginbottom{1cm}


%%%%%%%%%%%%%%%%%%%%%%%%%%%%%%%%%%%%%%%%%%%%%%%%%%%%%%%%%%%%%%%%%%%%%%%%%%%%%%%%
% FONTS
%%%%%%%%%%%%%%%%%%%%%%%%%%%%%%%%%%%%%%%%%%%%%%%%%%%%%%%%%%%%%%%%%%%%%%%%%%%%%%%%

\RequirePackage{fontspec}
\setmainfont{Carlito}


%%%%%%%%%%%%%%%%%%%%%%%%%%%%%%%%%%%%%%%%%%%%%%%%%%%%%%%%%%%%%%%%%%%%%%%%%%%%%%%%
% COLORS
%%%%%%%%%%%%%%%%%%%%%%%%%%%%%%%%%%%%%%%%%%%%%%%%%%%%%%%%%%%%%%%%%%%%%%%%%%%%%%%%

\colorlet{highlightbarcolor}{gter-gray}
\colorlet{headerbarcolor}{gter-green}
\colorlet{headerfontcolor}{gter-gray}
\colorlet{highlightbarfontcolor}{white}
\colorlet{accent}{gter-green}
\colorlet{heading}{black}
\colorlet{emphasis}{black}
\colorlet{body}{black}


%%%%%%%%%%%%%%%%%%%%%%%%%%%%%%%%%%%%%%%%%%%%%%%%%%%%%%%%%%%%%%%%%%%%%%%%%%%%%%%%
% set document
% non è obbligatorio riempire tutto, ad esempio se uno non ha github basta che
% commenta la riga
%%%%%%%%%%%%%%%%%%%%%%%%%%%%%%%%%%%%%%%%%%%%%%%%%%%%%%%%%%%%%%%%%%%%%%%%%%%%%%%%


\begin{document}
	
	% il contenuto di \tagline deve essere una versione breve del cv breve, non superare le 4 righe per evitare problemi di impaginazione (es. la dimensione della foto dipende dal numero di righe)
	
	\name{Simone Parmeggiani}
	\tagline{Geospatial Developer with expertise in Earth Observation Data Management\\ and ETL processes with geographic data. Open Source enthusiast}
	\photo[round]{picture.jpg}{\dimexpr \headerheight-\marginbottom}    %make photo exactly match the header with margintop/marginright/marginbottom as margin
	
	\makeheader
	
	\highlightbar{
		

		\section{Contact}

		
		\email{parmeggiani.simone@gmail.com}
		\phone{+39 348 2434565}
		\location{Via Alfred Bernhard Nobel, 6, Reggio Nell'Emilia}
		\vspace{0.5em}
		% \homepage{nicokrieger.com}{https://www.nicokrieger.de}
		\github{@SimoParmeg}{https://github.com/giantmolecularcloud}
		% \bitbucket{Nico Krieger}{https://www.linkedin.com/in/nico-krieger-6b28151b2/}
		\linkedin{Simone Parmeggiani}{https://www.linkedin.com/in/simone-parmeggiani/}
%		\orcid{0000-0003-1104-2014}{https://www.orcid.org/0000-0003-1104-2014}
%		\ads{NASA/ADS publication list}{https://ui.adsabs.harvard.edu/search/fq=\%7B!type\%3Daqp\%20v\%3D\%24fq\_database\%7D&fq\_database=database\%3A\%20astronomy&p\_=0&q=pubdate\%3A\%5B2016-01\%20TO\%209999-12\%5D\%20author\%3A(\%22Krieger\%2C\%20Nico\%22)&sort=date\%20desc\%2C\%20bibcode\%20desc}
		
		\section{Skills}
		
		\skillsection{Programming}
		\skill{Python}{5}
		\skill{SQL}{4}
		\skill{Javascript}{4}
		\skill{HTML/CSS}{4}
  		\skill{NoSQL / Elasticsearch}{3}
            \skill{C}{2}

		
		\vspace{0.5em}
		\skillsection{Operating Systems}
		\skill{Linux}{5}
		% \skill{MacOS}{5}
		\skill{Windows}{4}
		
		\vspace{0.5em}
		\skillsection{Software \& Tools}
            \skill{GIS Softwares}{5}
            \textcolor{highlightbarfontcolor}{(QGIS, ArcGIS)}\\
            \skill{DBMS}{5}
            \textcolor{highlightbarfontcolor}{(PostgreSQL, MySQL)}\\
		\skill{Data Visualization}{5}
		\textcolor{highlightbarfontcolor}{(e.g. matplotlib, gnuplot, ...)}\\
		\skill{Data handling/analysis}{5}
            \textcolor{highlightbarfontcolor}{(e.g. pandas, geopandas, numpy ...)}\\
            \skill{Geospatial Libraries}{5}
            \textcolor{highlightbarfontcolor}{(e.g. gdal, rasterio, netcdf4 ...)}\\
            \skill{QGIS Plugin}{5}
            \textcolor{highlightbarfontcolor}{(e.g. PyQGIS, PyQT ...)}\\
            \skill{Google Earth Engine}{5}
		\skill{Docker}{2}
		
		% \vspace{0.5em}
		% \skillsection{Another skill subsection header}
		% \simpleskill{You can also put simply text here without the dots.}
		
		\vspace{0.5em}
		\skillsection{Languages}
		% \skill{German}{5}
		\skill{English}{5}
		% \skill{French}{2}
		\bigskip
		
		\section{Certificates}
		\simpleskill{ESRI ARCGIS I, II, III, Spatial Analyst, Network Analyst}
		
		% \medskip
		% \section{Membership}
		% \simpleskill{AWS certified cloud practitioner}
		
	}
	\mainbar{
		\section{GIS Developer}
		Section are set in bold face. An optional parameter of \texttt{\textbackslash section} takes a symbol to add in front of the text. This option is used in the jobs and education sections below.
		
		\section[\faGears]{Work history}
		\job{Nov. 2021 - Oct. 2022}
		{JRC European Commission, Ispra (VA)}
		{Spatial Data Technology Expert}
		{Developed a cloud-based solution for Common Agriculture Policies monitoring}
  
		\job{July 2020 - Nov. 2021}
		{Hera SpA, Bologna(BO)}
		{Senior Data / GIS Analyst}
		{Asset Management data specialist}

  		\job{Sept 2019 - June 2020}
		{United Nations, Brindisi (BR)}
		{Senior Data / GIS Analyst}
		{Maintenance of the UN Geoportal}

  		\job{Aug 2018 - May 2019}
		{Here Technologies, Eindhoven, Netherlands}
		{GIS Analyst II}
		{Validation an Quality control of product data}

  %           \job{May 2018 - July 2018}
		% {Quix Srl, Modena (MO)}
		% {Software developer}
		% {Python backend developer}

            \job{May 2016 - May 2018}
		{Labs, Casalecchio di Reno (BO)}
		{GIS Analyst}

            \job{Dec 2013 - April 2016}
		{Here Technologies, Italy}
		{Map Data Collector}

		
		\section[\faMortarBoard]{Education}
		\job{01/2003 - 03/2012}
		{Alma Mater Studiorum, Bologna}
		{Urban Sociology}

  		\job{01/2013 - 12/2013}
		{University Tor Vergata, Rome}
		{Geographic Information Systems}

  		\job{01/2022 - 12/2022}
		{University of Harvard, CS50}
		{Introduction to Computer Science}

		
		% \section{Achievements, honours and awards}
		% \achievement{2013 - 1st Prize ESRI Foundation}

		
%		\section{General Skills}
%		\smallskip % additional skip because tag outlines use up space
%		\tag{Tag 1}
%		\tag{Tag 2}
%		\tag{and}
%		\tag{another tag}
%		\tag{some more tags}
%		\tag{yet another one}
%		\tag{tags flow over}
%		\tag{to the next line}
%		\tag{if necessary}
%		
%		\medskip
%		Tags must be ordered by hand with newlines to get a nice layout, especially for long tags.
		
		\section{Main activities}
		\bigskip
		% principali attività in gter tipo gis, webgis, formazione, r\&d, marketing, ecc.
		% per cambiare la percentuale modificare il numero iniziale es. 6/8
		% accent è l'identificativo del colore non modificarlo altrimenti sperisce il verde gter, il numero dopo il ! regola invece l'opacità del colore
		% This is taken from AltaCV
		% see https://github.com/liantze/AltaCV for details
		\wheelchart{2.2cm}{0.75cm}{% outer and inner diameter
			          % comma-separated list of
			8/8em/accent!80/webGIS,    % fraction of 24 / line length / color / label
			% 1/8em/accent!50/Marketing,          % here, the color is shades of the accent color
                3/8em/accent!60/Education,
			10/8em/accent/GIS Dev.,
			2/8em/accent!30/Research\&Development
		}
	}
	\makebody
	\clearpage
	
	
%	\pagestyle{highlightmain}
%	\highlightbar{}
%	\mainbar{
%		
%		\section{Another section}
%		
%		This page uses the page style \texttt{highlightmain} which shows the highlight bar (gray) and the main part (white background) but omits the header. 
%		The default page style is \texttt{headerhighlightmain} with all three elements.
%		If you don't want header, nor highlight bar, use page style \texttt{\textbackslash pagestyle\{empty\}}.
%		\medskip
%		Neither main, nor highlight bar must be filled to make this template work.
%		It is possible to use a page style with the highlight bar but leave it empty by setting an empty highlightbar \texttt{\textbackslash highlightbar\{\}}.
%		
%		\vspace{0.5em}
%		\subsection{Subsection 1}
%		Demonstrate subsections.
%		
%		\subsection{Subsection 2}
%		Subsection are also bold face but a smaller font then section. They also omit the rule.
%		
%		
%	}
%	\makebody
%	
%	
%	\clearpage
%	\pagestyle{empty}
%	
%	\section{Publications}
%	\pubforcefullwidth
%	
%	Demonstrate what an \texttt{\textbackslash pagestyle\{empty\}} page looks like.
%	Also show off the macros for publications that uses small icons for authors, date, journal and links.
%	
%	Achieving a good looking spacing can be tricky. For empty pagestyles where the full width is available use \texttt{\textbackslash pubforcefullwidth} to force the publoication list to take up all the available space.
%	The (relative) lengths reserved for date, journal and links can be set with the parameters \texttt{\textbackslash pubdatelength}, \texttt{\textbackslash pubjournallength} and \texttt{\textbackslash publinklength} as in \texttt{\textbackslash setlength\{\textbackslash pubdatelength\}\{0.15 \textbackslash linewidth\}}.
%	\bigskip
%	
%	\publication
%	{The turbulent gas structure in the centers of NGC~253 and the Milky Way} % Title
%	{\textbf{N. Krieger}, A. Bolatto, E. Koch, A. Leroy, E. Rosolowsky, F. Walter, A. Wei\ss, D. Eden, R. Levy, D. Meier, E. Mills, T. Moore, J. Ott, Y. Su, S. Veilleux} % Authors
%	{2020} % Year
%	{The Astrophysical Journal Vol. 899, Issue 2, id.158} % Journal
%	{\ADS{https://ui.adsabs.harvard.edu/abs/2020ApJ...899..158K}, \arXiv{https://arxiv.org/abs/2008.02518}} % ADS & arxiv links
%	
%	\publication
%	{The molecular ISM in the Super Star Clusters of the starburst NGC253} % Title
%	{\textbf{N. Krieger}, A. Bolatto, A. Leroy, R. Levy, E. Mills, D. Meier, S. Veilleux, F. Walter, A. Wei\ss} % Authors
%	{2020} % Year
%	{The Astrophysical Journal Vol.897, Issue 2, id.176} % Journal
%	{\ADS{https://ui.adsabs.harvard.edu/abs/2020ApJ...897..176K}, \arXiv{https://arxiv.org/abs/2006.08262}} % ADS & arxiv links
%	
%	\publication
%	{The Molecular Outflow in NGC\,253 at a Resolution of Two Parsecs} % Title
%	{\textbf{N. Krieger}, A. Bolatto, F. Walter, A. Leroy, L. Zschaechner, D. Meier, J. Ott, A. Wei\ss, E. Mills, S. Veilleux, M. Gorski} % Authors
%	{2019} % Year
%	{The Astrophysical Journal Vol.881, Issue 1, article id. 43, 20 pp} % Journal
%	{\ADS{https://ui.adsabs.harvard.edu/abs/2019ApJ...881...43K}, \arXiv{https://arxiv.org/abs/1907.00731}} % ADS & arxiv links
	
\end{document}